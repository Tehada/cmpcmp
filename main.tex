\documentclass{article}

\usepackage[utf8]{inputenc}
\usepackage[T2A]{fontenc}
\usepackage[russian, english]{babel}
\usepackage{listings} % исходный код
\usepackage{color}

\lstset{ % General setup for the package
    language=C++,
    basicstyle=\small\sffamily,
    numbers=left,
    numberstyle=\tiny,
    frame=tb,
    tabsize=4,
    columns=fixed,
    showstringspaces=false,
    showtabs=false,
    keepspaces,
    commentstyle=\color{red},
    keywordstyle=\color{blue}
}
\title{Решения задач по книге Дасгупта}
\author{Александр Шарипов}
\date{\today}

\begin{document}

\pagenumbering{gobble}
\maketitle
\newpage
\pagenumbering{arabic}

8 глава

\subparagraph{8.1} \textit{Покажите, что если поисковую задачу коммивояжёра можно решить за полиномиальное время, то и оптимизационный вариант можно решить за полиномиальное время.}

Пусть \textbf{n} -- кол-во вершин в графе, \textbf{A} -- матрица расстояний размера \textbf{[n x n]}. Считаем, что поисковую задачу мы умеем решать эффективно (за полиномиальное время). Сначала проверим, есть ли хоть один гамильтонов цикл для матрицы \textbf{A}, для этого найдём максимальный элемент \textbf{m} в матрице \textbf{A}, это займёт полиномиальное время -- \textbf{O(n^2)} и проверим, существует ли цикл веса не более \textbf{n * m}, проходящий по всем вершинам графа -- полиномиальное время по условию. Если не существует, то задача решена, если существует, то организуем бинарный поиск на отрезке \textbf{[n; n * m]} -- \textbf{log(n(m - 1))}.

\subparagraph{8.4} 



\subparagraph{8.5}

\subparagraph{8.8}

\subparagraph{8.9}

\subparagraph{8.12}

\subparagraph{8.13}

\subparagraph{8.16}

\subparagraph{8.17}

\subparagraph{8.20}

\subparagraph{8.21}

9 глава

\subparagraph{9.4}

\subparagraph{9.6}

\subparagraph{9.7}

\subparagraph{9.8}

\subparagraph{9.9}

\end{document}
