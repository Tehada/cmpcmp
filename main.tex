% Размер страницы и шрифта
\documentclass[12pt,a4paper]{article}

% Работа с русским языком

%\usepackage[T2A]{fontenc}            % кодировка
\usepackage{cmap}                     % поиск в PDF
\usepackage[T2A]{fontenc}             % кодировка
\usepackage[utf8]{inputenc}           % кодировка исходного текста
\usepackage[russian, english]{babel}  % локализация и переносы


% Размер полей
\usepackage[top=0.5in, bottom=0.75in, left=0.625in, right=0.625in]{geometry}

%\usepackage{titlesec}                % Изменение формата заголовков

\begin{document}

\section{8.1}

Пусть n -- кол-во вершин в графе, A -- матрица расстояний размера [n x n], в {} будем записывать сложностью проделанных операций.

Считаем, что поисковую задачу мы умеем решать эффективно (за полиномиальное время). Для начала поймём, есть ли хоть один гамильтонов цикл для матрицы A:

> "Найдём максимальный элемент в матрице A, обозначим его m" -- {O(n^2)}

> "Существует ли цикл веса не более n * m, проходящий по всем вершинам графа?" -- {полиномиальное время по условию}

Если не существует, то задача решена, если существует, то:

> "Организуем бинарный поиск на отрезке [n; n * m]" -- {log(n(m - 1))} (если m экспоненциально зависит от n, то поиск эффективен, если же двойная экспонента, то уже нет. problem!!).

\section{8.4},

\section{8.5}

\section{8.8}

\section{8.9}

\section{8.12}

\section{8.13}

\section{8.16}

\section{8.17}

\section{8.20}

\section{8.21}

\section{9.4}

\section{9.6}

\section{9.7}

\section{9.8}

\section{9.9}

\end{document}